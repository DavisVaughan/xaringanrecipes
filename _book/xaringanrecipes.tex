\documentclass[]{book}
\usepackage{lmodern}
\usepackage{amssymb,amsmath}
\usepackage{ifxetex,ifluatex}
\usepackage{fixltx2e} % provides \textsubscript
\ifnum 0\ifxetex 1\fi\ifluatex 1\fi=0 % if pdftex
  \usepackage[T1]{fontenc}
  \usepackage[utf8]{inputenc}
\else % if luatex or xelatex
  \ifxetex
    \usepackage{mathspec}
  \else
    \usepackage{fontspec}
  \fi
  \defaultfontfeatures{Ligatures=TeX,Scale=MatchLowercase}
\fi
% use upquote if available, for straight quotes in verbatim environments
\IfFileExists{upquote.sty}{\usepackage{upquote}}{}
% use microtype if available
\IfFileExists{microtype.sty}{%
\usepackage{microtype}
\UseMicrotypeSet[protrusion]{basicmath} % disable protrusion for tt fonts
}{}
\usepackage[margin=1in]{geometry}
\usepackage{hyperref}
\hypersetup{unicode=true,
            pdftitle={A Field Guide to xaringan},
            pdfauthor={Davis Vaughan},
            pdfborder={0 0 0},
            breaklinks=true}
\urlstyle{same}  % don't use monospace font for urls
\usepackage{natbib}
\bibliographystyle{apalike}
\usepackage{longtable,booktabs}
\usepackage{graphicx,grffile}
\makeatletter
\def\maxwidth{\ifdim\Gin@nat@width>\linewidth\linewidth\else\Gin@nat@width\fi}
\def\maxheight{\ifdim\Gin@nat@height>\textheight\textheight\else\Gin@nat@height\fi}
\makeatother
% Scale images if necessary, so that they will not overflow the page
% margins by default, and it is still possible to overwrite the defaults
% using explicit options in \includegraphics[width, height, ...]{}
\setkeys{Gin}{width=\maxwidth,height=\maxheight,keepaspectratio}
\IfFileExists{parskip.sty}{%
\usepackage{parskip}
}{% else
\setlength{\parindent}{0pt}
\setlength{\parskip}{6pt plus 2pt minus 1pt}
}
\setlength{\emergencystretch}{3em}  % prevent overfull lines
\providecommand{\tightlist}{%
  \setlength{\itemsep}{0pt}\setlength{\parskip}{0pt}}
\setcounter{secnumdepth}{5}
% Redefines (sub)paragraphs to behave more like sections
\ifx\paragraph\undefined\else
\let\oldparagraph\paragraph
\renewcommand{\paragraph}[1]{\oldparagraph{#1}\mbox{}}
\fi
\ifx\subparagraph\undefined\else
\let\oldsubparagraph\subparagraph
\renewcommand{\subparagraph}[1]{\oldsubparagraph{#1}\mbox{}}
\fi

%%% Use protect on footnotes to avoid problems with footnotes in titles
\let\rmarkdownfootnote\footnote%
\def\footnote{\protect\rmarkdownfootnote}

%%% Change title format to be more compact
\usepackage{titling}

% Create subtitle command for use in maketitle
\newcommand{\subtitle}[1]{
  \posttitle{
    \begin{center}\large#1\end{center}
    }
}

\setlength{\droptitle}{-2em}

  \title{A Field Guide to xaringan}
    \pretitle{\vspace{\droptitle}\centering\huge}
  \posttitle{\par}
    \author{Davis Vaughan}
    \preauthor{\centering\large\emph}
  \postauthor{\par}
      \predate{\centering\large\emph}
  \postdate{\par}
    \date{2018-05-05}

\usepackage{booktabs}

\usepackage{amsthm}
\newtheorem{theorem}{Theorem}[chapter]
\newtheorem{lemma}{Lemma}[chapter]
\newtheorem{corollary}{Corollary}[chapter]
\newtheorem{proposition}{Proposition}[chapter]
\newtheorem{conjecture}{Conjecture}[chapter]
\theoremstyle{definition}
\newtheorem{definition}{Definition}[chapter]
\theoremstyle{definition}
\newtheorem{example}{Example}[chapter]
\theoremstyle{definition}
\newtheorem{exercise}{Exercise}[chapter]
\theoremstyle{remark}
\newtheorem*{remark}{Remark}
\newtheorem*{solution}{Solution}
\begin{document}
\maketitle

{
\setcounter{tocdepth}{1}
\tableofcontents
}
\hypertarget{preface}{%
\chapter*{Preface}\label{preface}}
\addcontentsline{toc}{chapter}{Preface}

\href{https://github.com/yihui/xaringan}{\texttt{xaringan}} is a
powerful slide editor package built by
\href{https://github.com/yihui}{Yihui Xie}, a software engineer at
RStudio. While the defaults he sets are incredibly well thought out,
most of us can't resist wanting to tweak our slides in every way
imaginable. This book, inspired by
\href{https://twitter.com/hrbrmstr}{Bob Rudis}'s
\href{https://rud.is/books/21-recipes/index.html}{\emph{21 Recipes for
Mining Twitter with rtweet}}, is here to serve the purpose of providing
solutions to some of the most common tweaks that you might want to
perform. This will in no way be an exhaustive list, but hopefully can
get new users to \texttt{xaringan} up and running quickly.

\hypertarget{about-the-author}{%
\chapter*{About the Author}\label{about-the-author}}
\addcontentsline{toc}{chapter}{About the Author}

Davis Vaughan is a lover of all things R and Finance. He is graduating
with a Master's in Quantitative Finance in May 2018, with immediate
plans to work in the fixed income financial industry. He is the
co-author of several packages hosted under the
\href{http://www.business-science.io/}{Business Science} name including
\textbf{tidyquant}, \textbf{tibbletime}, and \textbf{timetk}.

\hypertarget{getting-started-with-xaringan}{%
\chapter{Getting started with
xaringan}\label{getting-started-with-xaringan}}

\hypertarget{problem}{%
\section{Problem}\label{problem}}

\hypertarget{solution}{%
\section{Solution}\label{solution}}

\hypertarget{discussion}{%
\section{Discussion}\label{discussion}}

\bibliography{book.bib,packages.bib}


\end{document}
