\documentclass[]{book}
\usepackage{lmodern}
\usepackage{amssymb,amsmath}
\usepackage{ifxetex,ifluatex}
\usepackage{fixltx2e} % provides \textsubscript
\ifnum 0\ifxetex 1\fi\ifluatex 1\fi=0 % if pdftex
  \usepackage[T1]{fontenc}
  \usepackage[utf8]{inputenc}
\else % if luatex or xelatex
  \ifxetex
    \usepackage{mathspec}
  \else
    \usepackage{fontspec}
  \fi
  \defaultfontfeatures{Ligatures=TeX,Scale=MatchLowercase}
\fi
% use upquote if available, for straight quotes in verbatim environments
\IfFileExists{upquote.sty}{\usepackage{upquote}}{}
% use microtype if available
\IfFileExists{microtype.sty}{%
\usepackage{microtype}
\UseMicrotypeSet[protrusion]{basicmath} % disable protrusion for tt fonts
}{}
\usepackage[margin=1in]{geometry}
\usepackage{hyperref}
\hypersetup{unicode=true,
            pdftitle={A Field Guide to xaringan},
            pdfauthor={Davis Vaughan},
            pdfborder={0 0 0},
            breaklinks=true}
\urlstyle{same}  % don't use monospace font for urls
\usepackage{natbib}
\bibliographystyle{apalike}
\usepackage{color}
\usepackage{fancyvrb}
\newcommand{\VerbBar}{|}
\newcommand{\VERB}{\Verb[commandchars=\\\{\}]}
\DefineVerbatimEnvironment{Highlighting}{Verbatim}{commandchars=\\\{\}}
% Add ',fontsize=\small' for more characters per line
\usepackage{framed}
\definecolor{shadecolor}{RGB}{248,248,248}
\newenvironment{Shaded}{\begin{snugshade}}{\end{snugshade}}
\newcommand{\AlertTok}[1]{\textcolor[rgb]{0.94,0.16,0.16}{#1}}
\newcommand{\AnnotationTok}[1]{\textcolor[rgb]{0.56,0.35,0.01}{\textbf{\textit{#1}}}}
\newcommand{\AttributeTok}[1]{\textcolor[rgb]{0.77,0.63,0.00}{#1}}
\newcommand{\BaseNTok}[1]{\textcolor[rgb]{0.00,0.00,0.81}{#1}}
\newcommand{\BuiltInTok}[1]{#1}
\newcommand{\CharTok}[1]{\textcolor[rgb]{0.31,0.60,0.02}{#1}}
\newcommand{\CommentTok}[1]{\textcolor[rgb]{0.56,0.35,0.01}{\textit{#1}}}
\newcommand{\CommentVarTok}[1]{\textcolor[rgb]{0.56,0.35,0.01}{\textbf{\textit{#1}}}}
\newcommand{\ConstantTok}[1]{\textcolor[rgb]{0.00,0.00,0.00}{#1}}
\newcommand{\ControlFlowTok}[1]{\textcolor[rgb]{0.13,0.29,0.53}{\textbf{#1}}}
\newcommand{\DataTypeTok}[1]{\textcolor[rgb]{0.13,0.29,0.53}{#1}}
\newcommand{\DecValTok}[1]{\textcolor[rgb]{0.00,0.00,0.81}{#1}}
\newcommand{\DocumentationTok}[1]{\textcolor[rgb]{0.56,0.35,0.01}{\textbf{\textit{#1}}}}
\newcommand{\ErrorTok}[1]{\textcolor[rgb]{0.64,0.00,0.00}{\textbf{#1}}}
\newcommand{\ExtensionTok}[1]{#1}
\newcommand{\FloatTok}[1]{\textcolor[rgb]{0.00,0.00,0.81}{#1}}
\newcommand{\FunctionTok}[1]{\textcolor[rgb]{0.00,0.00,0.00}{#1}}
\newcommand{\ImportTok}[1]{#1}
\newcommand{\InformationTok}[1]{\textcolor[rgb]{0.56,0.35,0.01}{\textbf{\textit{#1}}}}
\newcommand{\KeywordTok}[1]{\textcolor[rgb]{0.13,0.29,0.53}{\textbf{#1}}}
\newcommand{\NormalTok}[1]{#1}
\newcommand{\OperatorTok}[1]{\textcolor[rgb]{0.81,0.36,0.00}{\textbf{#1}}}
\newcommand{\OtherTok}[1]{\textcolor[rgb]{0.56,0.35,0.01}{#1}}
\newcommand{\PreprocessorTok}[1]{\textcolor[rgb]{0.56,0.35,0.01}{\textit{#1}}}
\newcommand{\RegionMarkerTok}[1]{#1}
\newcommand{\SpecialCharTok}[1]{\textcolor[rgb]{0.00,0.00,0.00}{#1}}
\newcommand{\SpecialStringTok}[1]{\textcolor[rgb]{0.31,0.60,0.02}{#1}}
\newcommand{\StringTok}[1]{\textcolor[rgb]{0.31,0.60,0.02}{#1}}
\newcommand{\VariableTok}[1]{\textcolor[rgb]{0.00,0.00,0.00}{#1}}
\newcommand{\VerbatimStringTok}[1]{\textcolor[rgb]{0.31,0.60,0.02}{#1}}
\newcommand{\WarningTok}[1]{\textcolor[rgb]{0.56,0.35,0.01}{\textbf{\textit{#1}}}}
\usepackage{longtable,booktabs}
\usepackage{graphicx,grffile}
\makeatletter
\def\maxwidth{\ifdim\Gin@nat@width>\linewidth\linewidth\else\Gin@nat@width\fi}
\def\maxheight{\ifdim\Gin@nat@height>\textheight\textheight\else\Gin@nat@height\fi}
\makeatother
% Scale images if necessary, so that they will not overflow the page
% margins by default, and it is still possible to overwrite the defaults
% using explicit options in \includegraphics[width, height, ...]{}
\setkeys{Gin}{width=\maxwidth,height=\maxheight,keepaspectratio}
\IfFileExists{parskip.sty}{%
\usepackage{parskip}
}{% else
\setlength{\parindent}{0pt}
\setlength{\parskip}{6pt plus 2pt minus 1pt}
}
\setlength{\emergencystretch}{3em}  % prevent overfull lines
\providecommand{\tightlist}{%
  \setlength{\itemsep}{0pt}\setlength{\parskip}{0pt}}
\setcounter{secnumdepth}{5}
% Redefines (sub)paragraphs to behave more like sections
\ifx\paragraph\undefined\else
\let\oldparagraph\paragraph
\renewcommand{\paragraph}[1]{\oldparagraph{#1}\mbox{}}
\fi
\ifx\subparagraph\undefined\else
\let\oldsubparagraph\subparagraph
\renewcommand{\subparagraph}[1]{\oldsubparagraph{#1}\mbox{}}
\fi

%%% Use protect on footnotes to avoid problems with footnotes in titles
\let\rmarkdownfootnote\footnote%
\def\footnote{\protect\rmarkdownfootnote}

%%% Change title format to be more compact
\usepackage{titling}

% Create subtitle command for use in maketitle
\newcommand{\subtitle}[1]{
  \posttitle{
    \begin{center}\large#1\end{center}
    }
}

\setlength{\droptitle}{-2em}

  \title{A Field Guide to xaringan}
    \pretitle{\vspace{\droptitle}\centering\huge}
  \posttitle{\par}
    \author{Davis Vaughan}
    \preauthor{\centering\large\emph}
  \postauthor{\par}
      \predate{\centering\large\emph}
  \postdate{\par}
    \date{2018-05-05}

\usepackage{booktabs}

\usepackage{amsthm}
\newtheorem{theorem}{Theorem}[chapter]
\newtheorem{lemma}{Lemma}[chapter]
\newtheorem{corollary}{Corollary}[chapter]
\newtheorem{proposition}{Proposition}[chapter]
\newtheorem{conjecture}{Conjecture}[chapter]
\theoremstyle{definition}
\newtheorem{definition}{Definition}[chapter]
\theoremstyle{definition}
\newtheorem{example}{Example}[chapter]
\theoremstyle{definition}
\newtheorem{exercise}{Exercise}[chapter]
\theoremstyle{remark}
\newtheorem*{remark}{Remark}
\newtheorem*{solution}{Solution}
\begin{document}
\maketitle

{
\setcounter{tocdepth}{1}
\tableofcontents
}
\hypertarget{preface}{%
\chapter*{Preface}\label{preface}}
\addcontentsline{toc}{chapter}{Preface}

\href{https://github.com/yihui/xaringan}{\texttt{xaringan}} is a
powerful slide editor package built by
\href{https://github.com/yihui}{Yihui Xie}, a software engineer at
RStudio. While the defaults he sets are incredibly well thought out,
most of us can't resist wanting to tweak our slides in every way
imaginable. This book, inspired by
\href{https://twitter.com/hrbrmstr}{Bob Rudis}'s
\href{https://rud.is/books/21-recipes/index.html}{\emph{21 Recipes for
Mining Twitter with rtweet}}, is here to serve the purpose of providing
solutions to some of the most common tweaks that you might want to
perform. This will in no way be an exhaustive list, but hopefully can
get new users to \texttt{xaringan} up and running quickly.

This book has a
\href{https://github.com/DavisVaughan/xaringanrecipes-companion}{companion
repo} made up of self-contained examples that demonstrate the topic of
each section. This way the reader can read about the information here,
and then go there to see all of the code. The rendered slides are also
linked to in each corresponding section.

Each section is broken into 4 subsections:

\begin{itemize}
\tightlist
\item
  Problem - Outline the problem to solve
\item
  Companion Deck - The rendered slide deck corresponding to the problem
\item
  Solution - A description of the solution to the problem
\item
  Discussion - Extra details and tips and tricks
\end{itemize}

\hypertarget{about-the-author}{%
\chapter*{About the Author}\label{about-the-author}}
\addcontentsline{toc}{chapter}{About the Author}

Davis Vaughan is a lover of all things R and Finance. He is graduating
with a Master's in Quantitative Finance in May 2018, with immediate
plans to work in the fixed income financial industry. He is the
co-author of several packages hosted under the
\href{http://www.business-science.io/}{Business Science} name including
\textbf{tidyquant}, \textbf{tibbletime}, and \textbf{timetk}.

\hypertarget{installing-xaringan}{%
\chapter{Installing xaringan}\label{installing-xaringan}}

\hypertarget{problem}{%
\section{Problem}\label{problem}}

You want to install the \texttt{xaringan} package.

\hypertarget{companion-deck}{%
\section{Companion Deck}\label{companion-deck}}

None

\hypertarget{solution}{%
\section{Solution}\label{solution}}

Great! You have two options, install from CRAN to get the stable
version, or from GitHub to get the development version.

\begin{Shaded}
\begin{Highlighting}[]
\KeywordTok{install.packages}\NormalTok{(}\StringTok{"xaringan"}\NormalTok{)}

\NormalTok{devtools}\OperatorTok{::}\KeywordTok{install_github}\NormalTok{(}\StringTok{"yihui/xaringan"}\NormalTok{)}
\end{Highlighting}
\end{Shaded}

\hypertarget{discussion}{%
\section{Discussion}\label{discussion}}

This book will use the CRAN version. Specifically version \texttt{0.6}.

\hypertarget{getting-started-with-xaringan}{%
\chapter{Getting started with
xaringan}\label{getting-started-with-xaringan}}

\hypertarget{problem-1}{%
\section{Problem}\label{problem-1}}

You want to start using \texttt{xaringan} to create slide decks in
RMarkdown.

\hypertarget{companion-deck-1}{%
\section{Companion Deck}\label{companion-deck-1}}

\href{http://xaringan-field-guide-companion.davisvaughan.com/02-getting-started/getting-started.html\#1}{slides}

\href{https://github.com/DavisVaughan/xaringanrecipes-companion/tree/master/02-getting-started}{deck
Rmd}

\hypertarget{solution-1}{%
\section{Solution}\label{solution-1}}

\texttt{xaringan} outputs a number of folders and files to contain
everything necessary to create your slides. Because of this, I suggest
using an RStudio Project to keep everything nice and tidy. From RStudio:

\texttt{File\ -\textgreater{}\ New\ Project...\ -\textgreater{}\ New\ Directory\ -\textgreater{}\ New\ Project}

Then create a \texttt{xaringan} slide deck from the template provided by
the package.

\texttt{File\ -\textgreater{}\ New\ File\ -\textgreater{}\ R\ Markdown\ -\textgreater{}\ From\ Template\ -\textgreater{}\ Ninja\ Presentation}

\hypertarget{discussion-1}{%
\section{Discussion}\label{discussion-1}}

The template is the easiest way to get started with \texttt{xaringan}.
It provides a full working example of the things you can build. As soon
as you open the template presentation, you can click \texttt{Knit} to
have it immediately render a presentation in the viewer.

\hypertarget{creating-a-slide}{%
\chapter{Creating a slide}\label{creating-a-slide}}

\hypertarget{problem-2}{%
\section{Problem}\label{problem-2}}

You want to create a slide in \texttt{xaringan}.

\hypertarget{companion-deck-2}{%
\section{Companion Deck}\label{companion-deck-2}}

\href{http://xaringan-field-guide-companion.davisvaughan.com/03-creating-a-slide/creating-a-slide.html\#1}{slides}

\href{https://github.com/DavisVaughan/xaringanrecipes-companion/tree/master/03-creating-a-slide}{deck
Rmd}

\hypertarget{solution-2}{%
\section{Solution}\label{solution-2}}

In \texttt{xaringan}, your entire presentation is created using 1
RMarkdown file.

\begin{itemize}
\item
  Slides are separated with a triple dash, \texttt{-\/-\/-}. The triple
  dash represents the beginning of a slide.
\item
  You can write plain text between two sets of \texttt{-\/-\/-} and have
  it show up as text on the slide.
\end{itemize}

The chunk below demonstrates how to create a slide, and add some text.

\begin{Shaded}
\begin{Highlighting}[]
\OperatorTok{---}

\NormalTok{Hello reader}

\OperatorTok{---}
\end{Highlighting}
\end{Shaded}

\hypertarget{discussion-2}{%
\section{Discussion}\label{discussion-2}}

There are two extra caveats here.

\begin{enumerate}
\def\labelenumi{\arabic{enumi})}
\item
  A title slide is automatically generated for you from the YAML header.
  This is the first chunk of information that you should see in the
  template. It should start with \texttt{title:}.
\item
  For the first slide after the title slide, you do not need to add a
  \texttt{-\/-\/-} to get it to render. It is already provided for you
  by the YAML header. The companion example for this lesson provides the
  following snippet to demonstrate this:
\end{enumerate}

\begin{Shaded}
\begin{Highlighting}[]
\ExtensionTok{---}
\ExtensionTok{title}\NormalTok{: }\StringTok{"Presentation Ninja"}
\ExtensionTok{subtitle}\NormalTok{: }\StringTok{"⚔<br/>with xaringan"}
\ExtensionTok{author}\NormalTok{: }\StringTok{"Yihui Xie"}
\ExtensionTok{date}\NormalTok{: }\StringTok{"2016/12/12"}
\ExtensionTok{output}\NormalTok{:}
  \ExtensionTok{xaringan}\NormalTok{::moon_reader:}
    \ExtensionTok{lib_dir}\NormalTok{: libs}
    \ExtensionTok{nature}\NormalTok{:}
      \ExtensionTok{highlightStyle}\NormalTok{: github}
      \ExtensionTok{highlightLines}\NormalTok{: true}
      \ExtensionTok{countIncrementalSlides}\NormalTok{: false}
\ExtensionTok{---}

\NormalTok{^ }\ExtensionTok{Everything}\NormalTok{ up until now was provided for you}
\ExtensionTok{This}\NormalTok{ is the slide following the title}

\ExtensionTok{---}

\ExtensionTok{Another}\NormalTok{ slide}
\end{Highlighting}
\end{Shaded}

\hypertarget{adding-r-chunks-to-slides}{%
\chapter{Adding R chunks to slides}\label{adding-r-chunks-to-slides}}

\hypertarget{problem-3}{%
\section{Problem}\label{problem-3}}

You want to add executable R code to a slide.

\hypertarget{companion-deck-3}{%
\section{Companion Deck}\label{companion-deck-3}}

\href{http://xaringan-field-guide-companion.davisvaughan.com/04-r-chunk/r-chunk.html\#1}{slides}

\href{https://github.com/DavisVaughan/xaringanrecipes-companion/tree/master/04-r-chunk}{deck
Rmd}

\hypertarget{solution-3}{%
\section{Solution}\label{solution-3}}

Just like with RMarkdown/Notebooks, you can add executable R code with
chunks. While on a slide, just add a chunk like you would with
RMarkdown, and type the code that you want to execute.

\begin{verbatim}
---

Here is some R code

```{r}
1 + 1
```

---
\end{verbatim}

This will render both the R code AND the output. We can use this to
programmatically generate images in our slides. By setting
\texttt{echo=FALSE}, only the image is shown on the slide.

\begin{verbatim}
---

We can use this to generate images programatically

```{r, echo=FALSE}
plot(cars)
```

---
\end{verbatim}

\hypertarget{discussion-3}{%
\section{Discussion}\label{discussion-3}}

\bibliography{book.bib,packages.bib}


\end{document}
